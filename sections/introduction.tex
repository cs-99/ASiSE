\section{Introduction}
\label{introduction}
Neuronal networks (NNs) are becoming increasingly relevant for industry and research. Their power stems from being able to approximate an arbitrary function from just input and output values through training and backpropagation. Therefore they are heavily used already today, for example in image recognition which can be used in medical applications or for autonomous systems such as automotives or roboters. Even in manufacturing they can be used productively, let it be for product design or quality inspection.\\ 
In recent research, analog NNs are occurring more and more frequently, as further improvement in general purpose processors slows down, while the demand for powerful NNs increases, slowly forcing research away from digital ones.\\
Even though over 30 years ago research has been conducted already in this topic \cite{Vittoz1990,Zurada1992,Graf1989,Harrer1992,Schwartz1989}, new developments and improvements are still being made with success. 
Therefore, this paper presents past and recent research in this topic to then summarize the current state of this research field and give a brief outlook into the possible future improvements.\\
Firstly though, the basic structure of NNs is explained to better understand the developments of the topic.